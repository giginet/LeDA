%伝家の宝刀今後の課題
\chapter{結論}

\section{結果}
本研究では、クラウドソーシングを用いてプレイログの収集、ログの閲覧を目的としたシステムを開発した。
ゲーム開発者の需要が概ね満たされ、開発者にとってはある程度有用なツールとなり得る。

\section{課題}
今回実装したシステムについて、現段階で考えられる問題点とその改善案について記した。


\subsection{SPAMの判定とフィルタリング}
今回開発したシステムを商用ゲーム開発の現場でも広く利用できるようにするためには、システムの信頼性を担保することが急務であると考えられる。

例えば、外部にアウトソーシングを行うに当たって、
まともにプレイしたプレイヤーか、収益のために適当なプレイを行ったプレイヤーかを判別、分類する機構が必要であると考えられる。

具体的には、連続して同じ操作が繰り返されているかどうかを検出する、同じレベルが何度もプレイされているが、一向に生存手数が伸びない、平均プレイ時間が極端に短い
などと言った、プレイヤーの質のスコア化が必要となってくるであろう。


\subsection{プレイヤー習熟度を考慮した統計処理}
現段階のシステムでは、データ的には、個々のプレイヤーを個別に追跡することが可能になっているが、
現在の統計処理は、個々のプレイヤー習熟度を一切考慮していないため、プレイヤー習熟度を考慮したシステム設計が必要だ。

まず、レベルの配信アルゴリズムについて、現在は完全なランダムで選択されているが、個々のプレイヤーの過去のプレイ状況から
配信される問題に重み付けをする方法が考えられる。

最も単純な方法は、すでにクリア済みのレベルは配信されるレベルから除外するという手法であろう。


また、レベルの推定された難易度と比較することで、このレベルをクリアできるプレイヤーであれば、これぐらいの熟練度を持っているというような
ヒューリスティックな習熟度の計測も可能であると考えられる。
そのため、この推定結果から、徐々に配信されるレベル難易度を上げていくなどの手法も考えられる。


\subsection{汎用的なシステムの模索}
今回提案したシステムは、対象としたゲームに特有のものであり、多くのゲーム作品に汎用的に利用可能なシステムにする方法を模索していくことも必要であろう。

第3章でも、さまざまなゲームに汎用的に利用可能なシステムのデータ構造などについて示したが、


具体的には、データの保存、取得を行うAPIを提供し、
ゲーム開発者がプレイログを収集したいタイミングで特定のコード片を埋め込むことで、汎用的にデータのやりとりを行わせる方法などが考えられる。

\subsection{レベルの改善点検出とその修正}
本研究の大目的であるレベルの改善点検出とその修正という目標は、未だ達成できていない。
収集したログを生かした、レベルの改善は今後の課題であろう。

どのゲームにでも応用しうる手法としては、
クリア率を単純に比較し、総プレイ回数とクリア率から、難易度を推定、難易度順に序列する方法だろう。
この場合、各種パラメータから、適切な難易度を推定する手法の考案が課題となる。

このゲームに特化した手法であれば、レベル中の各マスについて、到達状況をヒートマップ化し、クリアに不要な
マスを枝刈りするなどの方法が考えられる。
