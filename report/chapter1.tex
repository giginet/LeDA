%マジで書くことがないのでここが行数を稼ぎたい 目標:3枚
\chapter{はじめに}
\section{研究背景}
今日、日々、商用、非商用問わず多数のゲーム作品が世に生み出され続けている。
開発環境のオープン化と一般化が進み、個人でも簡単にゲーム作品を世に輩出することが可能になったことが一因としてあげられる。

同時に、世に出回るゲーム作品が増えたことで、商品として要求されるゲームの質、物量が一昔前と比べると、格段に肥大化し続けている。
重厚長大化するゲーム作品が求められることで、ゲーム開発はますます複雑化し、多数の人手、多額の資金、長期の開発期間、熟練された開発ノウハウが求められるようになってきた。

その一方で、ゲーム作品の単価は急速に低下しており、一作品数千円というパッケージ販売の時代から、一作品数百円や無料でプレイが可能なダウンロード主体の市場に変化しつつある。
そのため、ゲーム開発者は、如何に少ない労力で、高品質なゲームを生産する必要性に迫られている。

複雑化するゲーム開発の中でも、特にゲームのバランス調整や不具合検出と言ったゲームバランスの調整作業は多大な労力を要する。
ゲーム中に利用されているパラメータ群の数が肥大化し、最適な組み合わせを発見する作業が困難を極めるからだ。
ゲーム開発の現場において、ある特定のゲームルールにおいて、ゲームパラメータの調整を行い、
プレイヤーにとって適切な難易度を提供したり、ゲームルールの解説を行えるように設計する作業はレベルデザインと呼ばれている。
その中には、面クリア型のゲームにおけるステージ作成なども含まれる。
レベルデザインを高品質化することは、ゲーム作品自体のおもしろさ、ひいては売り上げにまで直結する。
質の高いレベルデザインを行うことはゲームを開発する意義そのものと直結すると言っても過言ではない。
しかし、レベルデザインを効果的に行う手法は未だ確立しておらず、調整作業は多数のテストプレイヤーを動員し、考え得るテストパターンを網羅的に行っているのが現状だ。

仮にこのような手法が発見されれば、ゲーム開発の生産性の向上に大きく寄与するであろう。

また、ここ数年、一般的に遊ばれるゲームの質の変容も著しい。
従来は、商用ゲーム業界においてはパッケージ型の販売が主流であった。そのため、一度販売し、リリース後にプレイヤーのフィードバックをゲーム自体に反映する作業は必要性に乏しく、次回作に生かされる程度であった。
それに対し、近年は世界的に、MMO\footnote{Massively Multiplayer Online-Game}と呼ばれる大規模オンラインゲームや、スマートフォン環境下でプレイ可能なソーシャルゲームへの市場の移行が進んでいる。
これらの形態のゲーム作品は、従来のパッケージ型の売切りのゲーム作品と異なり、リリース後もユーザーのプレイ状況やフィードバックを受け、継続的にサービスを改善、強化させ続けていくことが求められている。
そのため、ゲームパラメータの調整を、実際のプレイヤーログを集計して行う手法やシステムの重要度がより一層増している。

これらのゲーム中において、ユーザーのプレイログ収集は一般的に行われているが、それらを利用して効率的なレベルデザインを行う方法は未だ確立していない。


これらの理由から、プレイヤーログを用いた効率的なゲームの不具合の検出、 レベルデザインの改善手法の提案はゲーム開発者にとって、非常に有用な物と推測できる。

\section{研究目的}
本研究では、ゲーム開発者の需要である以下の点を満たす方法を模索する

\begin{itemize}
  \item 作成した「レベル」\footnote{ここでのレベルは、ゲーム開発特有の用語であり、第3章において詳解する}の難易度推定や改善点の検出、修正をある程度自動化して行いたい
  \item テストプレイヤーのプレイログを視覚的、かつ簡単に収集したい
  \item 収集したプレイログを統計的、視覚的に集約したい
\end{itemize}

また、最終的には、ゲームレベルの設計作業である、レベルデザインの改善、または作成したレベルの改善点検出、修正を効率的に行える手法を提案する。

今回は、レベルデザイン改善自動化の前段階として、
従来、多数のテストプレイヤーを動員する必要があったプレイログの収集をクラウド化し、収集、解析、可視化をWeb経由で簡便に行えるシステムの構築について模索する。

\section{論文構成}
本論文の残りの構成は下記の通りである。
\paragraph{第2章}
ゲームバランスの調整、ゲーム開発手法の提案、ゲームプレイヤーログの解析などと言った関連研究について触れ、これらの既存研究と本研究の差異、意義について考察する。
\paragraph{第3章}
研究目的を達成するために、今回提案する手法と仮説を提示する。また、プレイヤーログ収集システムの設計について解説する。
\paragraph{第4章}
実際にシステムを用いたプレイヤーログ収集について、システムの評価用に実装したゲームのルールを交えて解説する。
\paragraph{第5章}
本研究の結論を述べ、今後の課題を提示する。
