\chapter{関連研究}
\section{プレイログの収集に関する研究}
プレイヤーのプレイログの収集から、プレイヤーのゲームプレイを定性的に捉えようとした研究は多数存在する。
プレイヤーのプレイを集計し、ユーザビリティの向上を目指した研究\cite{metrics}\cite{usability}、
任天堂の「スーパーマリオブラザーズ」を元に作られた、オープンソースソフトウェア「Infinite Mario Bros」\cite{mario}を用い、プレイヤーのプレイログをモデル化する手法を提唱した研究\cite{model}\cite{optimization}など、
多数の先行研究が存在する。

\section{レベルデザインの改善に関する研究}
レベルデザインの改善をテーマにした研究としては「The 2010 Mario AI Championship Level Generation Track」\cite{level}が挙げられる。
これは先ほど挙げた「Infinite Mario Bros」を用いたコンテストであるMario AI Championship\cite{champion}の中で、
とりわけ、レベルの自動生成を目的とした部門、Level Generation Trackを題材とし、レベルデザインの自動化手法について多数紹介している。
また、複雑化したゲームパラメータの調整を最適化することを目標として掲げた研究として、「遺伝アルゴリズムの視覚化を用いたゲームのレベルデザイン効率化技法の開発」\cite{ga}が本研究と酷似している。
この研究では、複数のゲームパラメータの調整を最適化問題としてとらえ、遺伝的アルゴリズムを用いた手法の発見を模索した。

\section{ゲームの自動生成に関する研究}
レベルデザインのみならず、ゲームシステムそのものの自動生成を試みた研究も存在する。

特定の単語について、辞書データを元に適切なゲームを選び出し、当てはめる研究\cite{auto}や、ゲームルール自体を一から自動生成する研究\cite{gamedesign}などが挙げられる。


\section{既存研究との差異}
これらの既存研究と比較し、本研究においては以下の特徴が挙げられる。

\begin{itemize}
  \item プレイヤーへのゲームの配信、プレイ、評価、ログの集計など、一連の操作をクラウド上で行えるシステムを提案したこと
  \item このシステムを大規模に運用し、大量のプレイヤーログの収集を行ったこと
  \item 特定のゲーム作品だけではなく、様々なゲーム作品に汎化して適応可能なシステムを模索したこと
\end{itemize}
