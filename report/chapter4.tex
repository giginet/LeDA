\chapter{実験方法}
\section{対象とするゲーム}
\subsection{基本ルール}
今回、対象としたゲームは、ターン型完全情報のパズルゲームである。
このゲームは、10x10の盤面を操作し、キャラクターがゴール状態に辿り着くまでに繰り返す。
プレイヤーは一人でプレイする
\subsection{ターンの流れ}
まず、プレイヤーは盤面の中から隣接する2x2の任意の4マスについて選択する。
選択したマスについて、プレイヤーは選択したマスを右方向、ないしは左方向に回転させる。
ただし、何も無い空白のマスを含む4マスは回転させることができない。
マップを回転させると、上に乗っているオブジェクトやキャラクターも同様のルールによって回転する。
次に、プレイヤーキャラクターは、現在向いている方向に自動的に移動する。移動はスキップすることができない。
この一連の2つのステップを1ターンと定義する。プレイヤーはターンを終了状態に到達するまで繰り返し、
これを1ゲームとする。
\subsection{ゲームの終了条件}
ターンを繰り返し、いずれかの状態になったとき、1ゲームが終了する。

\subsubsection{クリア条件}
以下の状態になったとき、そのゲームはクリア状態となる
\begin{itemize}
  \item プレイヤーキャラクターがゴールマスに到達したとき
\end{itemize}

\subsubsection{ゲームオーバー条件}
以下の状態になったとき、そのゲームは失敗となり、ゲームオーバーになる
\begin{itemize}
    \item プレイヤーキャラクターがマップ外にはみ出したとき
    \item 死亡判定のあるギミックに触れたとき(穴や針など)
\end{itemize}

\subsection{マップ上のギミック}
マップ上のギミックとして、下記の物が実装されている。

\subsubsection{岩}
キャラクターは進入することができず、進行方向逆向きに方向を転換し、1マス進む。
進むべき方向に別の岩がある場合、方向のみを切り替える。
\subsubsection{穴}
プレイヤーキャラクターが進入するとゲームオーバーとなる。プレイヤー以外のキャラクターが進入すると消滅する。
\subsubsection{針}
キャラクターが4方向のうち、針の向いている方向から進入すると、穴と同様でゲームオーバーになる。
針のないその他の3方向から進入したとき、岩と同様の扱いとなる。
\subsubsection{滑る床}
プレイヤーキャラクターが進入すると、自動的に向いている方向にもう1マス進む。
進んだ後に進入した床がさら滑る床だった場合、
再帰的に進み続ける。
\subsubsection{ジャンプ台}
進入すると、向いている方向を1マス飛び越える。飛び越えたマスの効果は受けない。
\subsubsection{崩れる床}
1度目の進入は通常の床として扱う。崩れる床から他のマスに移動したとき、崩れる床は穴に変化する。以降は穴として扱う。
\subsubsection{方向パネル}
進入すると、現在の矢印の方向にキャラクターの向きを変える。

\section{対象ゲームの採用目的}
このゲームを実験対象のゲームとして採用したのは、以下の理由が挙げられる。
\subsubsection{操作が容易でプレイヤー間の実力差が出にくいため}
ルールさえ習得できれば、複雑な操作が必要なく、マウスのみで遊べるため、非常にプレイしやすい。
特殊な操作も必要のないため、先行研究で利用されていた『Infinite Mario』などに比べ、誰もが上達しやすいため。
\subsubsection{確定完全情報ゲームであるため}
ランダム性が一切ないため。ランダム要素を含むゲームだと、収集したログの解析が難しい物になってしまう。
また、ある手を取ったとき、次の状態が確定するため、計算機上でのシミュレーションも容易であるため、ログを収集した
データの定量的評価が行いやすい。
\subsubsection{リプレイ性が高いため}
様々な手を試し、試行錯誤をはかるゲームであるため、リプレイ性が高いと予想できる。
そのため、プレイ回数に応じたプレイヤー力量の変化を観測しやすい。

\subsubsection{問題空間が小さく、解の探索が容易なため}
プレイヤーの取れる行動が非常に少ないため、計算機上での全探索による解の探索、ソルバーの実装が容易である。

\subsubsection{レベル作成が非常に困難を極めるため}
レベルデータの作成は人の手によって行われているが、良質なレベルの作成が非常に困難である。
そのため、収集したプレイヤーログをレベル作成に生かすことができれば、非常に有用な物となるため。
レベル作成の困難性は後の項で詳しく解説する。


\section{対象ゲームの解の探索}
プレイヤーが取れる入力の総数Xは以下の数式で表される

\[ X = (2((W-1)・(H-1)))^T \]

ただし、W マップの横幅 H マップの高さ、Tは総ターン数

しかし、実際は何も無いマスを含む回転は行えないため、回転カ所は非常に限定される。


\section{対象ゲームのレベル設計方法}
\section{対象ゲームに特化したシステム設計}
